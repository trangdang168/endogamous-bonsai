\documentclass[12pt]{article}

\usepackage[top=0.5in, bottom=0.5in, left=0.5in, right=0.5in]{geometry}
\usepackage{authblk}
\usepackage{hyperref}
\usepackage[utf8]{inputenc}
\usepackage{amsmath}
\usepackage{amsfonts}
\usepackage{amssymb}
\usepackage{siunitx}
\usepackage{graphicx}
\usepackage{subcaption}
\usepackage{float}
\usepackage[nottoc,numbib]{tocbibind}
\usepackage{biblatex}

\bibliography{references.bib}

\title{Senior Thesis Proposal}
\author{Trang Dang}

\makeatletter
\let\inserttitle\@title
\let\insertauthor\@author
\makeatother

\begin{document}

\begin{center}
  \LARGE{\inserttitle}

  \Large{\insertauthor}
\end{center}

\section{Student Information}

Trang Dang - thdang@brynmawr.edu

\section{Summary}

% UNFINISHED, how much explaining background IBD and endogamy is there in here?

Here, we hope to adjust Bonsai\cite{Jewett2021} Background IBD finding algorithm that is originally designed for general populations to accomodate endogamous populations with higher levels of in-group marrianges. 

\section{Problem Statement}

Understanding inheritance in families is the first step towards understanding how rare genetics disorders can propagate within a family. A lot of these studies rely on Identical-By-Descent segments (IBD), which are segments of the DNA that descendants inherit from their ancestors. 

However, while there are multiple IBD-finding approaches, these approaches are designed for general populations, and, therefore, cannot handle endogamous populations, which are populations with lots of in-group marriages and little external admixture. 
For general populations with lots of genetic admixtures, usually, long matching sequences can only be inherited, and can be confidently declared IBD. 
However, as member of endogamous populations tend to marry within their communities, their descendants have many matches in their genetic sequences. 
Hence, when working with these populations, algorithms designed for general population tend to overrestimate the amount of IBD by confusing Identical-By-Descent with Identical-By-State segments, which are segments that people share because they are popular in the population. 

The Amish population, located in Lancaster, Pennsylvania, is an endogamous population with rates of Bipolar Disorder and Mood Disorder that are significantly higher than the US averages. We hope to explore how we can adapt Bonsai\cite{Jewett2021}, an algorithm that removes false positive IBD for general populations, for this Amish population. Hopefully, this can further contribute to understanding how traits are inherited in this community. 

% 19% bipolar, 33% mood (sara Mathieson_smbe_talk_final.pdf)

\section{Proposed Solution}

% more details
We will use relationships from pedigrees to identify the true IBD among the mix of IBD and IBS that algorithms designed for general population output. 
We will be working with a pedigree of 1338 individuals from the Amish population in Lancaster, PA. In this pedigree, 394 out of the 1338 individuals are genotyped\cite{Finke2021}. 
We have an approach, Bonsai\cite{Jewett2021}, that takes in two ancestors, their descentdants, IBDs, computes the probabilities, and decides to accept or reject the IBDs. We hope to first adjust some of the graph traversal in Bonsai so that this algorithm can handle multiple paths. 
Then, we hope to identify an approach to select the two ancestors that we can put into Bonsai.  

\section{Evaluation Plan}

We have a complete pedigree of 1338 Amish individuals, and 394 genotypes. We plan to use \texttt{ped-sim} to simulate IBDs from the genotypes and the pedigree. Then, we will use these simulations as the ground truth to evaluate the background IBDs that we've dropped.
To evaluate changes to the algorithm's implementation (such as replacing depth-first search 
with breadth first search), we will compare if the modified Bonsai or the original Bonsai return a set of IBD that most closely match the ground truth. 
To identify an approach that selects the best two sub-pedigrees to input into Bonsai, if we have several different criteria, we will observe how each of these criteria perform on small cases, and, if possible, we will evaluate how the outputs of each criteria compare against the ground truth. 

\section{Potential Challenges} 

We have multiple challenges because of the complex natures of pedigrees.

First, it's difficult working with graphs with cycles and many individuals. We can do case studies with small pedigrees of 10 individuals, but the full pedigree of more than 1000 individuals will contain different and complicated edge cases. We hope to alleviate this challenge by carefully study the relationships in the small pedigree. 

Secondly, modifying an existing implementation of an approach is extremely time-consuming. Therefore, we would try to be cautious. 

\section{Team Bios}

Trang Dang has been working on algorithms that works with Identical-By-Descent segments for 3 years in her research project, and this is one last direction she wants to investigate before graduating. She has worked with Bonsai before, so adjusting the algorithm sounds less intimidating. She also had experience regarding working with this Amish pedigree from a previous research project.

\printbibliography
\end{document}