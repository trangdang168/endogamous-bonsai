\documentclass[12pt]{article}

\usepackage[top=0.5in, bottom=0.5in, left=0.5in, right=0.5in]{geometry}
\usepackage{authblk}
\usepackage{hyperref}
\usepackage[utf8]{inputenc}
\usepackage{amsmath}
\usepackage{amsfonts}
\usepackage{amssymb}
\usepackage{siunitx}
\usepackage{graphicx}
\usepackage{subcaption}
\usepackage{float}
\usepackage[nottoc,numbib]{tocbibind}
\usepackage{biblatex}

\bibliography{references.bib}

\title{Senior Thesis Proposal}
\author{Trang Dang}

\makeatletter
\let\inserttitle\@title
\let\insertauthor\@author
\makeatother

\begin{document}

\begin{center}
  \LARGE{\inserttitle}

  \Large{\insertauthor}
\end{center}

\section{Student Information}

Trang Dang - thdang@brynmawr.edu

\section{Summary}


% UNFINISHED

Understanding inheritance in families is the first step towards understanding how rare genetics disorders can propagate within a family. A lot of these studies rely on Identical-By-Descent segments (IBD), which are segments of the DNA that descendants inherit from their ancestors. 

\section{Problem Statement}

Understanding inheritance in families is the first step towards understanding how rare genetics disorders can propagate within a family. A lot of these studies rely on Identical-By-Descent segments (IBD), which are segments of the DNA that descendants inherit from their ancestors. 

However, while there are multiple IBD-finding approaches, these approaches are designed for general populations, and, therefore, cannot handle endogamous populations, which are populations with lots of in-group marriages and little external admixture. 
For general populations with lots of genetic admixtures, usually, long matching sequences can only be inherited, and can be confidently declared IBD. 
However, as member of endogamous populations tend to marry within their communities, their descendants have many matches in their genetic sequences. 
Hence, when working with these populations, algorithms designed for general population tend to overrestimate the amount of IBD by confusing Identical-By-Descent with Identical-By-State segments, which are segments that people share because they are popular in the population. 

The Amish population, located in Lancaster, Pennsylvania, is an endogamous population with rates of Bipolar Disorder and Mood Disorder that are significantly higher than the US averages. We hope to design a method that can correctly identifies IBD segments for this population and contribute to investigating how the disorders are inherited in this population.

% 19% bipolar, 33% mood (sara Mathieson_smbe_talk_final.pdf)

\section{Proposed Solution}

% more details
We will use relationships from pedigrees to identify the true IBD among the mix of IBD and IBS that algorithms designed for general population output. 
We will be working with a pedigree of 1338 individuals from the Amish population in Lancaster, PA. In this pedigree, 394 out of the 1338 individuals are genotyped\cite{Finke2021}. 
We have an approach, Bonsai\cite{Jewett2021}, that takes in two ancestors, their descentdants, IBDs, computes the probabilities, and decides to accept or reject the IBDs. We hope to develop an algorithm that can select the ancestors and descentdants that will result in the most accurate Bonsai results.

% We will modify Bonsai's algorithm so that it can compute probabilites even when there are in-group marriages. (higher priority). Then, we will investigate 

\section{Evaluation Plan}

We have a complete pedigree of 1338 Amish individuals, and 394 genotypes. We plan to use \texttt{ped-sim} to simulate IBDs from the genotypes and the pedigree. Then, we will use these simulations as the ground truth to evaluate the IBDs that we've idenfied.

// what is the baseline for our evaluation?

\section{Potential Challenges} The first challenge for this project is adopting our probability calculations for Bonsai. We believe that if it's too challenging, we can set this task aside. The second challenge is the implementation: it can be difficult working with graphs with cycles and many individuals.

% modifies probability, modifies selecting pedigrees
% \section{Works We Built Upon}

% First, we will preprocess the genotype with \texttt{phasedibd}, an algorithm that identifies IBD for general populations\cite{Freyman2021}. 
% Then, we will use Bonsai, an algorithm that computes the likelihood that an observed IBD is an IBD and not an IBS. We will modify Bonsai's probabilities calculation using methods from our previous paper, \texttt{thread}, such that it accounts for how IBDs at different locations in the sequence are more or less likely inherited together\cite{Finke2021}. Finally, we will design our approach that selects the ancestors to input into the pedigree.

\section{Team Bios}

\printbibliography
\end{document}